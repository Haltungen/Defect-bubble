\documentclass[a4paper]{article}

%% Language and font encodings
\usepackage[english]{babel}
\usepackage[utf8x]{inputenc}
\usepackage[T1]{fontenc}
\usepackage{cite}

\usepackage{graphicx}
\usepackage{subcaption}
\usepackage{tikz}
\usepackage{pgfplots}
\usepackage{wrapfig}
\usepackage{cutwin}
\usetikzlibrary{intersections,decorations.pathreplacing,decorations.markings,calc}

%% Sets page size and margins
\usepackage[a4paper,top=3cm,bottom=2cm,left=2cm,right=2cm,marginparwidth=1.75cm]{geometry}

%% Useful packages
\usepackage{amsmath}
\usepackage{amsthm}
\usepackage{amssymb}
\usepackage{tikz}
\usepackage{graphicx}
\usepackage{hyperref}
\usepackage{caption}
%\usepackage{subcaption}
\usepackage{float}
\captionsetup{font={small,it}}
\theoremstyle{definition}
\newcommand{\D}{\mathbb{D}}
\newcommand{\Z}{\mathbb{Z}}
\newcommand{\R}{\mathbb{R}}
\newcommand{\C}{\mathcal{C}}
\newcommand{\A}{\mathcal{A}}
\newcommand{\B}{\mathcal{B}}
\newcommand{\dx}{\: \mathrm{d}}
\newcommand{\Scrystal}{\mathcal{S}_D^\#}
\newcommand{\KstarC}{(\mathcal{K}_D^{\#})^*}
\newcommand{\expl}[1]{\left[\text{\footnotesize \emph{#1}} \right]}
\newcommand{\ds}{\displaystyle}
\newcommand{\eqnref}[1]{(\ref {#1})}
\def\nm{\noalign{\medskip}}



\author{Erik Orvehed Hiltunen}

\begin{document}
\section{Problem statement and boundary integral formulation}
Assume that a single bubble occupies $D$, which is a circle of radius $R_b$ and center at the origin. Let $\C = \cup_{n\in\Z^2}(D+n)$ be the periodic bubble crystal.

Consider now a perturbed crystal, where $D$ is replaced by a defect circle $D_d$ of radius $R_d < R_b$. Let $\C_d = D_d \cup \left( \cup_{n\in\Z^2\setminus\{0,0\}} D+n \right)$ be the perturbed crystal. We consider the following problem
\begin{equation} \label{eq:scattering}
\left\{
\begin{array} {ll}
	&\ds \nabla \cdot \frac{1}{\rho} \nabla  u+ \frac{\omega^2}{\kappa} u  = 0 \quad \text{in} \quad \R^2 \backslash \C_d, \\
	\nm
	&\ds \nabla \cdot \frac{1}{\rho_b} \nabla  u+ \frac{\omega^2}{\kappa_b} u  = 0 \quad \text{in} \quad \C_d, \\
	\nm
	&\ds  u_{+} -u_{-}  =0   \quad \text{on} \quad \partial \C_d, \\
	\nm
	& \ds  \frac{1}{\rho} \frac{\partial u}{\partial \nu} \bigg|_{+} - \frac{1}{\rho_b} \frac{\partial u}{\partial \nu} \bigg|_{-} =0 \quad \text{on} \quad \partial \C_d
\end{array}
\right.
\end{equation}
Here, $\partial/\partial \nu$ denotes the outward normal derivative and $|_\pm$ denote the limits from outside and inside $D$.  

Let
\begin{equation*} % \label{data1}
v = \sqrt{\frac{\kappa}{\rho}}, \quad v_b = \sqrt{\frac{\kappa_b}{\rho_b}}, \quad k= \frac{\omega}{v} \quad \text{and} \quad k_b= \frac{\omega}{v_b}
\end{equation*}
be respectively the speed of sound outside and inside the bubbles, and the wavenumber outside and inside the bubbles. We also introduce two dimensionless contrast parameters
\begin{equation*} % \label{data2}
\delta = \frac{\rho_b}{\rho} \quad \text{and} \quad \tau= \frac{k_b}{k}= \frac{v}{v_b} =\sqrt{\frac{\rho_b \kappa}{\rho \kappa_b}}. 
\end{equation*}

Let $G(x,y)$ be the Green's function corresponding to the periodic crystal, i.e. $G$ satisfies
\begin{equation*} \label{eq:G}
\Delta G + (k^2+(k_b^2-k^2)\chi(\C))G = \delta(x-y)
\end{equation*}
Let $\mathcal{S}_{D}^k$ be the free-space single layer potential defined by
\begin{equation*}
\mathcal{S}_D^k[\phi](x) = \int_{\partial D} \Gamma(x,y)\phi(y) \dx \sigma(y), \quad x \in \R^2,
\end{equation*}
and let $\Scrystal$ be the single layer potential associated to the Green's function $G$, i.e.
\begin{equation*}
\Scrystal[\phi](x) = \int_{\partial D} G(x,y)\phi(y) \dx \sigma(y) , \quad x \in \R^2.
\end{equation*}

We seek a solution $u(x)$ of the form
\begin{equation*}
u(x) = \begin{cases}
\mathcal{S}_{D_d}^{k_b}[\phi_1](x) \quad &x\in D_d \\
\mathcal{S}_{D_d}^{k}[\phi_2](x) + S_D^k[\phi_3](x) & x\in D\setminus D_d \\
\Scrystal[\phi_4](x) & x \in \R^2\setminus D.
\end{cases}
\end{equation*}
A solution of this form satisfies the differential equation in \eqnref{eq:scattering}. The boundary conditions in equation \eqnref{eq:scattering} implies that the layer densities $\phi_i,\ i=1,2,3,4$ satisfies the system of boundary integral equations $\A(\omega, \delta)\Phi = 0$, where
\begin{equation*}
\A(\omega, \delta) = 
\begin{pmatrix}
\mathcal{S}_D^{k_b} &  -\mathcal{S}_D^{k} & -\mathcal{S}_{D_d,D}^{k} & 0 \\
0 & \mathcal{S}_{D,D_d}^k & \mathcal{S}_{D_d}^k & -\Scrystal \\
-\frac{1}{2}I+ \mathcal{K}_{D_d}^{k_b, *}& -\delta\left( \frac{1}{2}I+ (\mathcal{K}_D^{k})^*\right) & -\delta \frac{\partial \mathcal{S}_{D_d,D}^{k}}{\partial \nu} & 0 \\
0 & \frac{\partial \mathcal{S}_{D,D_d}^{k}}{\partial \nu} & -\frac{1}{2}I+ (\mathcal{K}_D^{k})^* & -\left( \frac{1}{2}I+ \KstarC\right)
\end{pmatrix}, 
\ \text{and}  \ \Phi= 
\begin{pmatrix}
\phi_1\\
\phi_2 \\
\phi_3 \\
\phi_4
\end{pmatrix}.
\end{equation*}
Here the operator $\mathcal{S}_{D_d,D}^{k} = \mathcal{S}_{D}^{k}|_{x\in \partial D_d}$ is the restriction of $\mathcal{S}_{D}^{k}$ onto $\partial D_d$.

\section{Numerical implementation}
We seek a spatial discretization of the boundary integral formulation. The factors $\Scrystal$ and $\KstarC$ require the Green's function $G$ for the crystal, so the equation \eqnref{eq:G} has to be solved numerically. We will apply the method found in \cite{bandgap}
Applying the Floquet transform we can decompose $G$ into the $\alpha$-quasiperiodic Green's function $G_\alpha$ which satisfies

\begin{equation*}
\Delta G_\alpha + (k^2+(k_b^2-k^2)\chi(\C))G_\alpha = \sum_{n\in \Z} \delta(x-y-n)e^{in\cdot\alpha}
\end{equation*}

Let $Y= [-1/2,1/2]^2$. For a fixed $y\in \R^2$, the function $u(x)=G_\alpha(x,y)$ is a solution to the problem
\begin{equation} \label{eq:quasiperiodic}
\left\{
\begin{array} {ll}
&\ds \nabla \cdot \frac{1}{\rho} \nabla  u+ \frac{\omega^2}{\kappa} u  = \delta(x-y) \quad \text{in} \quad Y \backslash D, \\
\nm
&\ds \nabla \cdot \frac{1}{\rho_b} \nabla  u+ \frac{\omega^2}{\kappa_b} u  = \delta(x-y) \quad \text{in} \quad D, \\
\nm
&\ds  u_{+} -u_{-}  =0   \quad \text{on} \quad \partial D, \\
\nm
& \ds  \frac{1}{\rho} \frac{\partial u}{\partial \nu} \bigg|_{+} - \frac{1}{\rho_b} \frac{\partial u}{\partial \nu} \bigg|_{-} =0 \quad \text{on} \quad \partial D \\
&  e^{-i \alpha \cdot x} u  \,\,\,  \text{is periodic.}
\end{array}
\right.
\end{equation}
Because of the reciprocity relation $G_\alpha(x,y) = G_\alpha(y,x)$, also $e^{-i \alpha \cdot y} G_\alpha(x,y)$ is periodic in $y$, so we can restrict to the case $y\in Y$. 
Then $G_\alpha$ can be written

\begin{equation*}
G_\alpha(x,y) = \begin{cases} \Gamma_\alpha^{k_b}(x,y) + S_D^{k_b}[\psi_b](x) \quad &x\in D \\  \Gamma_\alpha^{k}(x,y) + S_D^{k}[\psi](x) &x\in Y \backslash \bar{D} \end{cases}
\end{equation*}
Using the jump relations for the single layer potentials, we find that
\begin{equation*}
\B(\omega,\delta)[\Psi] = F
\end{equation*}
where 
\begin{equation*}
\B(\omega, \delta) = 
\begin{pmatrix}
\mathcal{S}_D^{k_b} &  -\mathcal{S}_D^{\alpha,k}  \\
-\frac{1}{2}+ \mathcal{K}_D^{k_b, *}& -\delta( \frac{1}{2}+ (\mathcal{K}_D^{ -\alpha,k})^*)
\end{pmatrix}, 
\,\, \Psi= 
\begin{pmatrix}
\psi_b\\
\psi
\end{pmatrix},
\,\, F=
\begin{pmatrix}
\Gamma_\alpha^{k} - \Gamma_\alpha^{k_b} \\
\delta\frac{\partial \Gamma_\alpha^{k}}{\partial \nu} -
\frac{\partial \Gamma_\alpha^{k_b}}{\partial \nu} 
\end{pmatrix}
\end{equation*}

Recall that the quasi-periodic Green's function $\Gamma_\alpha^k$, defined as the solution to the equation
\begin{equation*}
\Delta \Gamma_\alpha^k(x,y) + k^2\Gamma_\alpha\alpha^k(x,y) = \sum_{n\in \Z} \delta(x-y-n)e^{in\cdot\alpha}
\end{equation*}
can be expanded as 
\begin{equation} \label{eq:quasihomogenious}
\Gamma_\alpha^k(x,y) = -\frac{i}{4}\sum_{m\in \Z^2} H_0^{(1)}(k|x-y-m|)e^{im\cdot\alpha}
\end{equation}
We need to expand the function $\Gamma_\alpha^k(x,y)$ in terms of the polar coordinates $(r,\theta)$ of $x$. We will use the following versions of Graf's addition theorem.

\begin{equation*}
H_l^{(1)}(kr_2)e^{il\theta_2} =
\begin{cases}
\sum_{n=-\infty}^\infty H_{l-n}^{(1)}(kb)e^{i(l-n)\beta}J_n(kr_1)e^{in\theta_1} \qquad &\text{if } r_1<b \\
\sum_{n=-\infty}^\infty H_{l-n}^{(1)}(kr_1)e^{i(l-n)\theta_1}J_n(kb)e^{in\beta} \qquad &\text{if } r_1>b
\end{cases}
\end{equation*}
In these equations we have $x_1 = r_1e^{i\theta_1}, x_2 = r_2e^{i\theta_2}$ and $x_2 = x_1 + be^{i\theta}$. 

In the following, pick $x$ on the boundary $\partial D$, i.e. $x = R_be^{i\theta}$. Furthermore, pick $y=r'e^{i\theta'}$ inside $Y$. Using the addition formulas, we have
\begin{align*}
H_0^{(1)}(k|x-y-m|) &= 
\begin{cases}
\sum_{n=-\infty}^\infty (-1)^nH_{-n}^{(1)}(k|y+m|)e^{-in\theta_m'}J_n(kR_b)e^{in\theta} \qquad &\text{if } R_b<|y+m| \\
\sum_{n=-\infty}^\infty (-1)^nH_{-n}^{(1)}(kR_b)e^{-in\theta}J_n(k|y+m|)e^{in\theta_m'} \qquad &\text{if } R_b>|y+m|
\end{cases}
\end{align*}
For $m\neq 0$ we have $R_b<|y+m|$ and
\begin{equation*}
H_{-n}^{(1)}(k|y+m|)e^{-in\theta_m'} = \sum_{l=-\infty}^\infty H_{-n-l}^{(1)}(k|m|)e^{i(-n-l)\theta_m}J_l(kr')e^{il\theta'}
\end{equation*}
Plugging in above expressions into equation \ref{eq:quasihomogenious}, we find
\begin{equation*}
\Gamma_\alpha^k(x,y) = -\frac{i}{4}\sum_{n=-\infty}^\infty\left[ M_ne^{in\theta} + \sum_{l=-\infty}^\infty\left[ \sum_{m\in \Z^2, m\neq 0} H_{-n-l}(k|m|)e^{i(-n-l)\theta_m}e^{im\cdot\alpha} \right] (-1)^nJ_l(kr')e^{il\theta'}J_n(kR_b)e^{in\theta}\right],
\end{equation*}
where the terms $M_n$, corresponding to $m=0$, are given by
\begin{equation*}
M_n = \begin{cases}
(-1)^nH_{-n}(kr')e^{-in\theta'}J_n(kR_b) \quad &\text{if } r' > R_b \\
(-1)^nH_{n}(kR_b)J_{-n}(kr')e^{-in\theta'} \quad &\text{if } r' < R_b.
\end{cases}
\end{equation*}
The two different cases correspond to the source $y$ being inside or outside the bubble. Define the lattice sum $Q_n$ as 
\begin{equation*}
Q_n = \sum_{m\in \Z^2, m\neq 0} H_{n}(k|m|)e^{in\theta_m}e^{im\cdot\alpha}.
\end{equation*}
Then the equation for $\Gamma_\alpha^k$ is 
\begin{equation}\label{eq:gammafourier}
\Gamma_\alpha^k(x,y) = -\frac{i}{4}\sum_{n=-\infty}^\infty\left[ M_n + \sum_{l=-\infty}^\infty Q_{-n-l} (-1)^nJ_l(kr')e^{il\theta'}J_n(kR_b)\right]e^{in\theta}.
\end{equation}
This can be viewed as a Fourier series expansion of $\Gamma_\alpha^k$ as a function of $x\in S^1$. The $n$:th Fourier coefficient is
\begin{equation*}
-\frac{i}{4}\left[M_n + \sum_{l=-\infty}^\infty Q_{-n-l} (-1)^nJ_l(kr')e^{il\theta'}J_n(kR_b)\right]
\end{equation*}
For $x\in \partial D$ we have
\begin{equation*}
\frac{\partial \Gamma_\alpha^k}{\partial \nu(x)} = \frac{\partial \Gamma_\alpha^k}{\partial r}
\end{equation*} 
Differentiating equation \ref{eq:gammafourier} we find
\begin{equation*}
\frac{\partial \Gamma_\alpha^k}{\partial \nu(x)} = -\frac{i}{4}\sum_{n=-\infty}^\infty\left[ M_n' + \sum_{l=-\infty}^\infty Q_{-n-l} (-1)^nJ_l(kr')e^{il\theta'}kJ_n'(kR_b)\right]e^{in\theta},
\end{equation*}
where
\begin{equation*}
M_n' = \begin{cases}
(-1)^nkH_{-n}(kr')e^{-in\theta'}J_n'(kR_b) \quad &\text{if } r' > R_b \\
(-1)^nkH'_{n}(kR_b)J_{-n}(kr')e^{-in\theta'} \quad &\text{if } r' < R_b.
\end{cases}
\end{equation*}


\bibliography{defect}{}
\bibliographystyle{plain}
\end{document}
