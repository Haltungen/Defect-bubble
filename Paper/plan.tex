\documentclass[a4paper]{article}

%% Language and font encodings
\usepackage[english]{babel}
\usepackage[utf8x]{inputenc}
\usepackage[T1]{fontenc}
\usepackage{cite}

\usepackage{graphicx}
\usepackage{subcaption}
\usepackage{tikz}
\usepackage{pgfplots}
\usepackage{wrapfig}
\usepackage{cutwin}
\graphicspath{ {images/} }
\usetikzlibrary{intersections,decorations.pathreplacing,decorations.markings,calc}

%% Sets page size and margins
\usepackage[a4paper,top=3cm,bottom=2cm,left=2.5cm,right=2.5cm,marginparwidth=1.75cm]{geometry}

%% Useful packages
\usepackage{amsmath}
\usepackage{amsthm}
\usepackage{amssymb}
\newtheorem{prop}{Propostition}
\usepackage{tikz}
\usepackage{graphicx}
\usepackage{hyperref}
\usepackage{caption}
%\usepackage{subcaption}
\usepackage{float}
\captionsetup{font={small,it}}
\theoremstyle{definition}
\newcommand{\D}{\mathbb{D}}
\newcommand{\Z}{\mathbb{Z}}
\newcommand{\R}{\mathbb{R}}
\newcommand{\C}{\mathcal{C}}
\newcommand{\A}{\mathcal{A}}
\newcommand{\B}{\mathcal{B}}
\newcommand{\Sc}{\mathcal{S}}
\newcommand{\K}{\mathcal{K}}
\newcommand{\dx}{\: \mathrm{d}}
\newcommand{\Scrystal}{\mathcal{S}_D^\#}
\newcommand{\KstarC}{(\mathcal{K}_D^{\#})^*}
\newcommand{\expl}[1]{\left[\text{\footnotesize \emph{#1}} \right]}
\newcommand{\ds}{\displaystyle}
\newcommand{\eqnref}[1]{(\ref {#1})}
\def\nm{\noalign{\medskip}}

\title{Localized modes for acoustic waves in a bubbly crystal with a defect}
\author{\small Author: Erik Orvehed Hiltunen, Master student in Engineering physics, Uppsala University \\ \small Supervisor: Habib Ammari, Professor of Applied Mathematics, Department of Mathematics, ETH Zürich}

\begin{document}
\maketitle
\section{Purpose and goals}
We consider acoustic waves propagation through a liquid with bubbles. These bubbles resonates at a frequency which correspond to wavelengths much larger than the bubble, the so called Minnaert resonance frequency. Specifically, we consider a periodic array of bubbles, where the size of one of the bubbles is perturbed. The goal is to analytically and numerically show that this crystal has a localized eigenmode close to the defect bubble.

It has previously been shown that an unperturbed, periodic bubble crystal has a bandgap close to the Minnaert resonance \cite{bandgap}. Physically, this means that waves with frequencies inside the bandgap will be exponentially decaying inside the crystal. Furthermore, it has been experimentally shown that a perturbed crystal will have a localized mode, i.e. for waves of certain frequencies inside the bandgap, the wave will be localized around the perturbation and will be exponentially decaying away from the perturbation \cite{experiment}. 

\section{Planning}
zThe following plan shows the work which needs to be done and the time plan for the project. 

\begin{table}[h]
	\centering
\begin{tabular}{r|p{13cm}}
	Week & Planned work \\ \hline
	2017 - 37 & Reading old papers, especially \cite{first} and \cite{bandgap}\\
	38 & Reading old papers, especially \cite{first} and \cite{bandgap}\\
	39 & Working on the essential spectrum for the original and perturbed crystals\\
	40 & Continue work on essential spectrum, formulating the perturbed problem\\
	41 & Formulating the perturbed problem and the boundary integral equations \\
	42 & Solving the model problem with two different methods \\
	43 & Working on equations for the numerical implementation, Taylor expansions of the quasi-periodic Green's function\\
	44 & Starting the numerical implementation\\
	45 & Continuing with the numerical implementation, starting writing the paper\\
	46 & Deriving the correct jump conditions for crystal single layer potential, debugging the code\\
	47 & Finish the code, start the analysis. Read on Taylor expansions for single layer potentials for perturbed boundaries.\\
	48 & Use Taylor expansions to reduce the 4x4 system to a 2x2 system\\
	49 & Perform asymptotic analysis for the radius to write the system as the original + perturbation\\
	50 & Show that the system has a characteristic value inside the bandgap\\
	51 & Finish the analysis\\
	2018 - 03 & Write the paper\\
	04 & Write the paper\\
	05 & Write the paper\\
	06 & Finish the paper\\
	
\end{tabular}
\end{table}
\bibliography{defect}{}
\bibliographystyle{plain}
\end{document}
